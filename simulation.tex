\documentclass[12pt]{article}
\usepackage{polski}
\usepackage[utf8]{inputenc}
\usepackage[polish]{babel}
\DeclareUnicodeCharacter{00A0}{ }
\title{Czy żyjemy w symulacji komputerowej?}
\author{autor: Nick Bostrom}
\author{tłumaczenie: Jacek Karwowski}

\begin{document}
	\maketitle
	\begin{abstract}
		Poniższa praca dowodzi, że przynajmniej jedna z trzech następujących tez jest prawdziwa: (1) gatunek ludzki prawdopodobnie wyginie przed osiągnięciem fazy "postludzkiej"; (2) istnieje bardzo niska szansa, że jakakolwiek postludzka cywilizacja prowadziłaby znaczącą liczbę symulacji historii swojej ewolucji lub wariacji tejże; (3) prawie na pewno żyjemy wewnątrz symulacji komputerowej. Wynika z tego, że jeżeli nie jesteśmy w takiej sytuacji, to prawdopodobnie nasza szansa osiągnięcia fazy postludzkiej i stanie uruchomienia własnej symulacji jest nikła. Praca ta omawia także inne konsekwencje takiego stanu.
	\end{abstract}
	\section{Wstęp}
		Znacząca liczba publikacji zarówno poważnych techno- i futurologów, jak i autorów science-fiction przewiduje, że w przyszłości ludzkość będzie miała dostęp do niewyobrażalnych zasobów mocy obliczeniowej. Załóżmy na ten moment, że te przewidywania są prawdziwe. Jedną z rzeczy, do których przyszłe pokolenia mogą użyć swoich potężnych superkomputerów, są szczegółowe symulacje życia ich przodków, ewentualnie ludzi podobnych do ich przodków. Ponieważ komputery staną się tak niezwykle potężne, będą oni mogli pozwolić sobie na uruchamianie dużej ilości takich symulacji. Przypuśćmy, że ci symulowani ludzie są świadomi (stałoby się tak tylko gdyby symulacje byłyby wystarczająco szczegółowe oraz jeżeli pewna dość szeroko akceptowana teza z dziedziny filozofii umysłu jest rzeczywiście prawdziwa). W takim przypadku istnieje możliwość, że znacząca większość umysłów takich jak nasze nie należy do oryginalnej rasy, lecz jest symulowana przez odległych potomków "pierwotnych" ludzi. Racjonalnym byłoby wtedy stwierdzenie, że my sami należymy raczej do symulowanych, nie zaś pierwotnych i "czysto biologicznych" ludzi. Kontynuując, jeżeli nie zgadzamy się z myślą, że żyjemy w symulacji komputerowej, to nie powinniśmy także sądzić, że kiedykolwiek będziemy w stanie stworzyć znaczącą ilość takowych. To jest idea leżąca u podstaw mojej pracy. Pozostała część publikacji jest poświęcona na dokładniejsze i ściślejsze sprecyzowanie tej myśli. \\
		Oprócz zaciekawienia ludzi zaangażowanych w spekulacje futurologiczne, publikacja ta ma dostarczyć także materiału do rozważań bardziej teoretycznych. Zastosowana argumentacja może stać się bodźcem do sformułowania pewnych metodologicznych oraz metafizycznych pytań, a także pokazuje naturalistyczne analogie do tradycyjnych koncepcji religijnych\textbf{, co być może przynajmniej niektórzy uznają za interesujące lub nawet stymulujące intelektualnie}.\\
		Struktura pracy przedstawia się następująco: najpierw sformułujemy założenia z dziedziny filozofii umysłu, których potrzebujemy do przedstawienia naszej argumentacji. Następnie zajmujemy się rozważaniami natury empirycznej, które dostarczają pewnych argumentów za tym, że przeprowadzenie symulacji opisywanych w pracy leży w zasięgu dostatecznie rozwiniętej cywilizacji, spełnia ograniczenia stawiane przez współczesną inżynierię,oraz nie łamie żadnych praw fizycznych, które w tym momencie znamy. Ta część nie jest konieczna do zrozumienia części głównej - argumentacji filozoficznej - ale może być intelektualnie pobudzające.
		Część następna jest rdzeniem dowodu. Korzystam w niej z pewnych prostych metod rachunku prawdopodobieństwa, a także pokazuję argumenty przemawiające za uznaniem słabej zasady nierozróżnialności \footnote{ang. weak indifference principle}. Wywód kończę omawiając możliwe interpretacje alternatywy przedstawionej w abstrakcie, co jest jednocześnie podsumowaniem całej pracy.
	\section{Założenie niezależności od bazy} 
		Cos \footnote{ang. assumption of substrate-independence}
\end{document}